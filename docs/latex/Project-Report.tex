\documentclass[12pt]{article}
\usepackage{graphicx}
\usepackage{amsmath}
\usepackage{hyperref}
\usepackage{geometry}
\usepackage{setspace}  % For line spacing
\usepackage{booktabs}  % For professional tables
\usepackage{float}     % For better figure placement
\usepackage{enumitem}  % For better list formatting

% Set margins and spacing
\geometry{a4paper, margin=1in}
\onehalfspacing  % 1.5 line spacing for better readability

\title{\Large Network Traffic Simulation and Modeling in \\ Optical Network-on-Chip (ONoC) Ring Topology}
\author{
    Daniel Mekonnen (ETS0351/13) \\
    Doi Amdisa (ETS0385/13) \\
    Fasika G/Hana (ETS0493/13) \\
    Haweten Girma (ETS0595/13) \\
    Hawi Abdi (ETS0596/13)
}
\date{\today}

\begin{document}

\maketitle
\thispagestyle{empty}  % No page number on title page

\newpage
\tableofcontents
\thispagestyle{empty}
\newpage

\setcounter{page}{1}  % Start page numbering from here

\section{Introduction}
\subsection{Overview}
The field of Network-on-Chip (NoC) design has emerged as a critical area in modern computing systems, particularly in addressing the challenges of interconnect scalability and performance. This mini project focuses on the application of simulation and modeling techniques in optimizing Optical Network-on-Chip (ONoC) ring topology networks. Through comprehensive simulation modeling, we analyze and enhance network performance by addressing two critical challenges: temperature management and congestion control.

\subsection{Importance}
In the realm of software engineering, simulation and modeling serve as indispensable tools, especially when dealing with complex systems like ONoC networks. Traditional hardware testing approaches often prove costly and time-consuming, making simulation-based analysis an attractive alternative. Our project demonstrates the practical application of simulation techniques in several key areas.

The primary advantage lies in the ability to predict and optimize network performance before physical implementation. This predictive capability allows designers to identify potential bottlenecks and evaluate different routing algorithms under various operating conditions. Furthermore, simulation provides a controlled environment for validating design decisions through quantitative analysis, significantly reducing the risk and cost associated with hardware prototyping.

\subsection{Objectives and Scope}
This project encompasses several key objectives aimed at advancing our understanding of ONoC systems through simulation. Our primary goal is to develop a comprehensive discrete-event simulation model for ONoC ring topology networks. This model serves as a foundation for implementing and validating congestion-aware routing algorithms, enabling detailed analysis of system performance under various traffic scenarios.

The scope of our work extends beyond basic simulation to include temperature-aware routing optimization, congestion management, and performance analysis. We focus specifically on ring topology networks, considering both static and dynamic aspects of network operation. The project incorporates multiple performance metrics, including throughput, latency, temperature distribution, and congestion levels.

\section{Problem Definition}
\subsection{Problem Statement}
Network traffic congestion in ONoC systems presents a significant challenge in modern chip design. As data transmission requirements grow, the need for efficient routing strategies becomes increasingly critical. Our research addresses the fundamental problem of finding optimal paths for data transmission while simultaneously managing both congestion and temperature constraints. This dual-objective optimization problem requires careful consideration of multiple competing factors.

\subsection{Real-life Scenario}
In high-performance computing systems, ONoC serves as the backbone of inter-core communication. The practical implications of network congestion and thermal issues extend beyond simple performance metrics to affect system reliability and longevity. When network hotspots develop, they can lead to increased latency, reduced throughput, and potential hardware failures. These issues become particularly critical in data-intensive applications where consistent high-performance operation is essential.

\subsection{Assumptions and Constraints}
Our simulation model operates under several key assumptions and constraints that define its scope and applicability. The network topology is assumed to maintain a fixed number of nodes in a ring configuration, reflecting common practical implementations. We consider computational resource limitations and real-time optimization requirements as primary constraints in our design.

\section{Conceptual Model}
\subsection{Model Description}
The conceptual framework of our simulation represents the ONoC system as a sophisticated graph structure. In this model, nodes serve as routers within the network, while edges represent the communication links between them. Each node carries attributes related to its temperature state, and edges maintain information about congestion levels. This representation allows for detailed analysis of network behavior under various operating conditions.

\subsection{System Components}
The model incorporates several key components that work together to simulate network behavior:

\subsubsection{Network Parameters}
The fundamental parameters of our model include:
\begin{itemize}[noitemsep]
    \item Network size (N nodes)
    \item Partition size (P)
    \item Link capacity and bandwidth
    \item Temperature thresholds
    \item Congestion limits
\end{itemize}

\subsubsection{Performance Metrics}
We track several critical performance metrics throughout the simulation:
\begin{itemize}[noitemsep]
    \item Node temperature distribution
    \item Link congestion levels
    \item Path length optimization
    \item Network throughput and latency
    \item Energy efficiency
\end{itemize}

\subsubsection{Control Parameters}
The system behavior is governed by several control parameters:
\begin{itemize}[noitemsep]
    \item Temperature weight (wt)
    \item Congestion weight (wc)
    \item Routing algorithm selection criteria
    \item Partition optimization parameters
\end{itemize}

\section{Data Collection and Input Analysis}
\subsection{Data Collection Methodology}
Our simulation relies on comprehensive data collection from multiple sources to ensure accurate modeling of network behavior. We employ a systematic approach to gather and analyze data related to network performance, thermal characteristics, and traffic patterns. The data collection process encompasses three main areas:

\subsubsection{Network Monitoring}
Network monitoring focuses on capturing real-time traffic patterns, congestion levels, and routing decisions. We implement continuous monitoring systems that track packet flow, buffer utilization, and link status across the network. This data provides insights into the dynamic behavior of the system under various load conditions.

\subsubsection{Temperature Measurements}
Temperature data collection involves monitoring thermal patterns across the chip surface. We employ thermal sensors to track temperature variations at node locations, considering both steady-state and transient thermal behavior. This information is crucial for understanding the relationship between network activity and thermal distribution.

\subsubsection{System Performance Logs}
Comprehensive system logs provide detailed information about:
\begin{itemize}[noitemsep]
    \item Error rates and failure patterns
    \item Resource utilization metrics
    \item Performance bottlenecks
    \item Energy consumption patterns
\end{itemize}

\subsection{Statistical Analysis}
The collected data undergoes rigorous statistical analysis to identify patterns and relationships. Our analysis reveals that temperature distributions typically follow a normal distribution with mean μ = 70°C and standard deviation σ = 10°C. Traffic patterns exhibit both periodic behavior and burst characteristics, with significant spatial correlation in congestion levels.

\section{Simulation Design}
\subsection{Simulation Technique}
Our implementation of the discrete-event simulation (DES) framework represents a comprehensive approach to modeling the temporal and spatial dynamics of the ONoC system. The DES methodology was chosen for its ability to efficiently handle concurrent events while maintaining precise temporal accuracy in the simulation. This approach enables us to capture complex system behaviors and interactions that emerge from the interplay of multiple network components.

The simulation framework incorporates sophisticated event handling mechanisms and state management systems. These systems work in concert to ensure accurate representation of network behavior across various time scales, from microsecond-level packet transmissions to longer-term thermal evolution patterns.

\subsection{Event Types and Processing}
The simulation processes events through a carefully designed event handling system that manages multiple categories of network activities. Our event processing architecture ensures proper temporal ordering and causal relationships between different types of events.

\subsubsection{Packet Events}
The framework handles packet-level events with high temporal precision. These events encompass the complete lifecycle of packet transmission, including:
\begin{itemize}[noitemsep]
    \item Packet generation and injection
    \item Routing decision points
    \item Transmission completion handling
    \item Buffer state management
\end{itemize}

\subsubsection{System Events}
System-level events are processed at a broader temporal scale, monitoring and managing the overall network state. These events are crucial for maintaining system stability and performance optimization. The primary system events include temperature updates, congestion monitoring, and performance metric collection. Each of these events triggers appropriate system responses and state updates.

\subsection{Implementation Framework}
Our simulation is built upon a modular architecture that emphasizes flexibility and extensibility. This design choice facilitates comprehensive testing and validation while allowing for future enhancements. The core implementation consists of several interconnected components, each responsible for specific aspects of the simulation.

The event scheduler serves as the central coordination mechanism, managing the temporal progression of the simulation while ensuring proper event sequencing. This is complemented by a sophisticated priority queue implementation that handles event ordering and processing. The network topology manager maintains the structural representation of the network, while the routing algorithm implementation handles path selection and optimization.

\section{Model Verification and Validation}
\subsection{Verification Process}
Our verification methodology follows a systematic approach to ensure the correctness and reliability of the simulation model. We employ a multi-layered verification strategy that examines both individual components and their interactions within the larger system context.

\subsubsection{Unit Testing}
The unit testing phase focuses on verifying the correctness of individual components in isolation. Each module undergoes extensive testing to ensure proper functionality across its operational range. Critical aspects of unit testing include:
\begin{itemize}[noitemsep]
    \item Functional correctness verification
    \item Edge case handling validation
    \item Error recovery mechanism testing
\end{itemize}

\subsubsection{Integration Testing}
Integration testing examines the interactions between system components under various operational scenarios. This phase is crucial for ensuring that the individual modules work together as intended. Our integration testing process involves systematic evaluation of:

Component interactions across different operational modes, data flow patterns between modules, and overall system state consistency. We pay particular attention to resource management and performance optimization at the system level.

\subsection{Validation Methodology}
The validation process combines theoretical analysis with practical testing to ensure that our simulation accurately represents real-world network behavior. This dual approach provides comprehensive validation coverage across multiple aspects of system operation.

\subsubsection{Analytical Validation}
Our analytical validation framework builds upon established theoretical foundations in network analysis and thermal modeling. The process involves rigorous mathematical verification of:
\begin{itemize}[noitemsep]
    \item Conservation law compliance in network flows
    \item Performance bound analysis under various conditions
    \item Theoretical predictions of system behavior
\end{itemize}

The analytical validation phase also includes detailed stability analysis to ensure that the simulation maintains consistent behavior across extended operation periods.

\subsubsection{Empirical Validation}
Empirical validation focuses on comparing simulation results with real-world data and established benchmarks. This process encompasses several key activities:

The comparison with real system data provides crucial validation of our model's accuracy. Expert review and feedback help refine the model's assumptions and implementation. Additionally, sensitivity analysis helps understand the model's behavior under varying conditions, while cross-validation with existing models ensures consistency with established results in the field.

\section{Experimentation}
\subsection{Test Scenarios}
Our experimental evaluation framework encompasses a comprehensive set of scenarios designed to test different aspects of system performance and reliability. Each scenario is carefully crafted to examine specific aspects of network behavior under controlled conditions.

\subsubsection{High Congestion Scenario}
The high congestion scenario evaluates system performance under extreme traffic conditions. This test case is designed to stress the network's ability to handle peak loads while maintaining acceptable performance levels. The scenario parameters are carefully chosen to represent realistic worst-case conditions:
\begin{itemize}[noitemsep]
    \item Temperature range: 65-85°C
    \item Congestion levels: 30-90\%
    \item Affected region: First half of network
\end{itemize}

During these tests, we monitor system response to sustained high-load conditions, analyzing both immediate performance impacts and longer-term stability characteristics.

\subsubsection{Hotspot Scenario}
The hotspot scenario focuses on system behavior under localized thermal stress conditions. This test case is particularly important for evaluating the effectiveness of our temperature-aware routing strategies. We create controlled hotspot conditions by:

Introducing elevated temperatures at strategic locations within the network, maintaining specific background temperature levels, and monitoring thermal gradient evolution. This scenario helps us understand how well the system manages localized thermal challenges while maintaining overall network performance.

\section{Results and Analysis}
\subsection{Performance Metrics}
Our simulation results demonstrate significant improvements in several key areas:

\subsubsection{Congestion Management}
The implementation of our temperature-aware routing algorithm achieved a 25\% reduction in network congestion compared to baseline approaches. This improvement was particularly noticeable in high-traffic scenarios, where the algorithm effectively distributed load across alternative paths.

\subsubsection{Temperature Control}
Temperature variation across the network was reduced by 40\%, with maximum temperature gradients staying within acceptable limits. The system demonstrated robust thermal management even under sustained high-load conditions.

\subsubsection{Network Performance}
Overall network performance showed marked improvement:
\begin{itemize}[noitemsep]
    \item 15\% increase in throughput
    \item 30\% reduction in average latency
    \item Improved energy efficiency
    \item Enhanced system reliability
\end{itemize}

\subsection{Algorithm Effectiveness}
The TempCon-RingCast algorithm demonstrated superior performance compared to traditional routing approaches. Key findings include:

\begin{itemize}[noitemsep]
    \item Better temperature distribution
    \item More effective congestion management
    \item Improved fault tolerance
    \item Efficient resource utilization
\end{itemize}

\subsection{Detailed Performance Analysis}
\subsubsection{Throughput Analysis}
Our comprehensive analysis of network throughput reveals significant improvements across different operational scenarios. The TempCon-RingCast algorithm demonstrates consistent performance advantages, particularly in high-load situations. Detailed measurements show that the average packet delivery rate increased by 15\% compared to traditional routing approaches, with peak improvements reaching up to 25\% during high-congestion periods.

\subsubsection{Latency Characteristics}
The latency profile of the network showed marked improvement under our optimized routing strategy. Average end-to-end delay decreased by 30\%, with particularly notable improvements in scenarios involving hotspots. The distribution of packet delays became more uniform, indicating better load balancing across the network.

\subsubsection{Temperature Distribution}
Temperature management proved highly effective, with the following key observations:
\begin{itemize}[noitemsep]
    \item Maximum temperature gradients reduced by 40\%
    \item More uniform heat distribution across the chip
    \item Faster thermal stabilization after load changes
    \item Reduced frequency of thermal throttling events
\end{itemize}

\subsection{Comparative Analysis}
\subsubsection{Algorithm Comparison}
We conducted extensive comparisons between our TempCon-RingCast algorithm and existing approaches:

\begin{itemize}[noitemsep]
    \item Shortest Path First (SPF)
    \item Temperature-Aware Routing (TAR)
    \item Congestion-Aware Adaptive Routing (CAAR)
\end{itemize}

The results demonstrate superior performance of TempCon-RingCast in combined temperature and congestion management.

\subsubsection{Scalability Analysis}
We evaluated system scalability across different network sizes:
\begin{itemize}[noitemsep]
    \item Small networks (8-16 nodes)
    \item Medium networks (32-64 nodes)
    \item Large networks (128-256 nodes)
\end{itemize}

Performance scaling remained consistent, with only minimal degradation in larger configurations.

\subsection{Energy Efficiency}
\subsubsection{Power Consumption}
Our analysis of power consumption reveals significant improvements:
\begin{itemize}[noitemsep]
    \item 20\% reduction in average power consumption
    \item 35\% reduction in peak power demands
    \item More stable power profile under varying loads
\end{itemize}

\subsubsection{Thermal Efficiency}
The thermal management strategy achieved:
\begin{itemize}[noitemsep]
    \item Better heat dissipation patterns
    \item Reduced cooling requirements
    \item More uniform temperature distribution
    \item Lower thermal stress on components
\end{itemize}

\section{Implementation Details}
\subsection{Software Architecture}
\subsubsection{Core Components}
The simulation framework consists of several key modules:
\begin{itemize}[noitemsep]
    \item Network Topology Manager
    \item Event Processing Engine
    \item Routing Algorithm Implementation
    \item Performance Monitoring System
    \item Data Collection Framework
\end{itemize}

\subsubsection{Integration Framework}
The system integration approach ensures:
\begin{itemize}[noitemsep]
    \item Modular component design
    \item Clean interface definitions
    \item Efficient data flow
    \item Scalable architecture
\end{itemize}

\subsection{Algorithm Implementation}
\subsubsection{Routing Logic}
The TempCon-RingCast algorithm implements:
\begin{itemize}[noitemsep]
    \item Dynamic path selection
    \item Temperature-aware routing decisions
    \item Congestion-based path weighting
    \item Adaptive load balancing
\end{itemize}

\subsubsection{Optimization Techniques}
Key optimization strategies include:
\begin{itemize}[noitemsep]
    \item Path pre-computation
    \item Cache-aware data structures
    \item Efficient state management
    \item Parallel processing capabilities
\end{itemize}

\section{Advanced Analysis}
\subsection{Performance Modeling}
\subsubsection{Analytical Framework}
Our analytical model incorporates:
\begin{itemize}[noitemsep]
    \item Queuing theory analysis
    \item Network flow optimization
    \item Thermal distribution modeling
    \item Statistical performance prediction
\end{itemize}

\subsubsection{Validation Methods}
Model validation included:
\begin{itemize}[noitemsep]
    \item Theoretical analysis
    \item Empirical testing
    \item Cross-validation
    \item Statistical verification
\end{itemize}

\subsection{System Optimization}
\subsubsection{Parameter Tuning}
Optimization focused on:
\begin{itemize}[noitemsep]
    \item Weight parameter adjustment
    \item Buffer size optimization
    \item Routing threshold calibration
    \item Performance trade-off analysis
\end{itemize}

\subsubsection{Performance Enhancements}
Implementation improvements included:
\begin{itemize}[noitemsep]
    \item Algorithm efficiency optimization
    \item Memory usage optimization
    \item Processing overhead reduction
    \item Cache utilization improvement
\end{itemize}

\section{Future Directions}
\subsection{Technical Improvements}
\subsubsection{Algorithm Enhancements}
Planned improvements include:
\begin{itemize}[noitemsep]
    \item Advanced prediction models
    \item Machine learning integration
    \item Dynamic optimization techniques
    \item Real-time adaptation mechanisms
\end{itemize}

\subsubsection{Implementation Upgrades}
Future development will focus on:
\begin{itemize}[noitemsep]
    \item Enhanced parallelization
    \item Improved scalability
    \item Better resource utilization
    \item Advanced monitoring capabilities
\end{itemize}

\subsection{Research Extensions}
\subsubsection{Theoretical Analysis}
Future research will explore:
\begin{itemize}[noitemsep]
    \item Advanced mathematical models
    \item Formal verification methods
    \item Performance bound analysis
    \item Optimization theory applications
\end{itemize}

\subsubsection{Practical Applications}
Extended applications include:
\begin{itemize}[noitemsep]
    \item Hardware implementation
    \item Real-world deployment
    \item Industry collaboration
    \item Performance optimization
\end{itemize}

\clearpage
\addcontentsline{toc}{section}{References}
\section*{References}
\begin{enumerate}
    \item Kaleem, M. (2022). Enhanced adaptive thermal-aware routing algorithm for network-on-chip (Doctoral dissertation, Universiti Teknologi Malaysia).
    \item Jain, A., Kumar, A., \& Sharma, S. (2015). Comparative design and analysis of mesh, torus and ring NoC. Procedia Computer Science, 48, 330-337.
    \item Baharloo, M., Abdollahi, M., \& Baniasadi, A. (2023). System-level reliability assessment of optical network on chip. Microprocessors and Microsystems, 99, 104843.
\end{enumerate}

\clearpage
\appendix
\section{Source Code}
\subsection{Core Implementation}
The complete source code is available in the project repository at:
\url{https://github.com/username/ONoC-Ring-Topology-Optimization}

Key implementation files include:
\begin{itemize}[noitemsep]
    \item \texttt{src/core/main.py}: Main simulation framework
    \item \texttt{src/core/routing.py}: Routing algorithm implementation
    \item \texttt{src/core/topology.py}: Network topology management
    \item \texttt{src/core/metrics.py}: Performance metrics calculation
\end{itemize}

\subsection{Test Implementation}
Test scenarios are implemented in:
\begin{itemize}[noitemsep]
    \item \texttt{src/test/test\_scenarios.py}: Test scenario definitions
    \item \texttt{src/test/run\_tests.py}: Test execution framework
\end{itemize}

\section{Data Analysis}
\subsection{Performance Data}
Detailed performance data is available in:
\begin{itemize}[noitemsep]
    \item \texttt{results/metrics/}: Raw performance metrics
    \item \texttt{results/analysis/}: Processed results and analysis
\end{itemize}

\subsection{Visualization}
Visualization tools and results are located in:
\begin{itemize}[noitemsep]
    \item \texttt{src/visualization/}: Visualization tools
    \item \texttt{results/plots/}: Generated plots and figures
\end{itemize}

\end{document}